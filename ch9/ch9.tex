\documentclass[12pt]{amsart}

\addtolength{\hoffset}{-2.25cm}
\addtolength{\textwidth}{4.5cm}
\addtolength{\voffset}{-2.5cm}
\addtolength{\textheight}{5cm}
\setlength{\parskip}{0pt}
\setlength{\parindent}{15pt}

\usepackage{amsthm}
\usepackage{amsmath}
\usepackage{amssymb}
\usepackage[colorlinks = true, linkcolor = black, citecolor = black, final]{hyperref}

\usepackage{graphicx}
\usepackage{multicol}
\usepackage{ marvosym }
\usepackage{wasysym}
\newcommand{\ds}{\displaystyle}


\pagestyle{myheadings}

\setlength{\parindent}{0in}

\pagestyle{empty}

\begin{document}

\thispagestyle{empty}

{\scshape VETM 690} \hfill {\scshape \Large  Ch. 9: Artifical Selection} \hfill {\scshape Feb 28 2021}
\medskip
\hrule
\bigskip

\begin{itemize}
\item Artificial selection (AS) is the main agent for the genetic improvement of plants and animals
\item Process: Use only the most suitable individuals as reproducers in relation to a specific objective
\item Also a tool for experimental research: can alter a phenotypic trait permanently and this response to selection is from changes in allele frequencies at the loci affecting that trait
\item The initial response to AS can be used to estimate the genetic variance of the selected trait in the population. It also produces changes in correlated traits, allowing the estimation of genetic covariances between different characters (or the same char in diff. environments)
\item From long term response we can infer the number of genes involved and their initial frequencies (and possible interactions between natural and artificial selection)
\item Selection in a population lacking genetic variation can enable the estimation of mutation rate, effects, dominance relationships, and pleiotropic effects 
\item Application: on phenotypes, but response can be improved when integrating information from relatives and molecular genetic markers
\item Response to selection (and its prediction):
\begin{itemize}
\item Consider the selection of a trait based on phenotype assuming the infinitesimal model: the distribution of phenotypic values is normal with mean $\bar{X}$ and variance $\sigma_p^2$ 
\item Measure the response to selection as the change in population mean after one generation
\item regress average values of a character trait for pairs of parents onto those of their offspring
\item Slope of this regression line is the estimate of heritability 
\item truncation selection: just select the parents with the highest phenotypic values, and the mean of the trait value of these parents deviated from the overall mean is the selection differential (S) 
\item The mean of the offspring from selected parents (deviated from the population mean) is the response to selection R
\item We get the expression $R = h^2S$ known as the breeder's equation
\item Regression to mediocrity: the average value for a trait of selected offspring is lower than that of their selected parents, and this is because heritability is always less than 1, with some of the superiority of the parents being attributable to environmental influences
\item The smaller number of parents used, the greater the selection differential 
\item The response to selection can be predicted without doing the selection if estimates on heritability and phenotypic variance are available (and assuming normality)
\item Standardize $R/\sigma_P = h^2(S/\sigma_P)$ leads to the expression for response to selection $R = ih\sigma_A$ where i is the standardized selection differential and $\sigma_A^2$ is the additive variance for the trait
\item Because of the normal distribution, the average value of selected individuals can be expressed as $i = z/\phi$, where z is the value of the ordinate and phi is the truncation point for a selected proportion
\item But in a population with finite census size, selection intensity is lower than that from the expression above (caveat: inbreeding generated by selection implies a reduction in long-term response)
\item Sex considerations: must be averaged across sex, or if only one sex is considered then $i = i_f/2$
\item From the standardized selection differential, we see it is key to select a low proportion of individuals as parents (applied: few or 1 sire and many dams), or alternatively we can reduce the phenotypic variance of the population (by either reducing environmental variation or experimental error) 
\item Overlapping generations: present the response per year $R_a = R/I_g$ where $I_g$ is the generation interval 
\end{itemize}
\item Correlated responses
\begin{itemize}
\item We can see correlated responses when the trait that we select for is correlated with other traits ($r_A$), resulting in an expression for coheritability which is $R_{CY} = i_Xr_Ah_Xh_Y\sigma_{PY}$
\end{itemize}
\item Measure of the response
\begin{itemize}
\item Extending the response to selection over a few generations, the change in mean follows a linear trajectory
\item Means between generations are subject to change from chance, measurement errors, environmental factors or the genetic composition of the population (but assuming randomness these should not produce a bias)
\item A bias could be produced from sustained environmental changes or adaptation by natural selection. To avoid: maintain control lines without selection
\item If the response is linear, we can quantify it as the regression of the mean of each generation on the ordinate, creating a descriptive statistic for the summary of response but not the magnitude of applied selection
\item To incorporate the magnitude of selection, we calculate the total response as the regression coefficient of the cumulative response on the selection differential accumulated over generations (termed realized heritability $h^2_r$  
\item If several lines are available to apply selection, then we can obtain estimates for the empirical variance of selection response. Recall that the variance of line means for an additive trait (neutral model) is approximated by $tV_A/N_e$ when $t << 2N_e$, but the variance of measurement errors is $V_P/n$, so we can compute the expected value for the variance of the phenotypic response $E[V(R)] = V_P(\frac{th^2}{N_e}+\frac{1}{n}$, useful for getting the sampling variance of the realized heritability 
\item Standard errors of estimated realized heritability are obtainable from the standard error of the regression but this is biased downwards since the means of different generations are correlated
\item But estimations of realized heritability from REML approaches accounts for this correlation, accounts for fixed effects, genetic changes and dominance effects
\item Epistatic variation: realized heritability and selection response are functions of additive x additive components, so that the realized heritability becomes $h^2_r = (V_a + 0.5V_{AA})/V_P$ ignoring higher order interaction terms
\end{itemize}
\item Asymmetry of response
\begin{itemize}
\item Responses to selection can be asymmetric due to genetic drift (especially with low $N_e$), genetic basis of the character (whether it is due to few loci of large effect, if allele frequencies are not intermediate, or dominance), due to inbreeding depression (reducing the response effect, ID can be partially eliminated using an unselected line of the same census size as the selected line as a control), maternal effects, changes in phenotypic variance, and opposition of natural selection to change in the mean of one direction of selection but not the other
\item Selection that lowers fecundity or increases mortality rates of offspring will negatively bias selection differentials, detectable by calculating selection differentials weighing value of parents by contribution to offspring
\item asymmetry most often occurs when a selected trait is component of fitness since the response to decrease the mean is accompanied by inbreeding depression 
\item Furthermore, deleterious alleles that are at low frequencies in the initial population will show larger responses to selection
\end{itemize}
\item Change of allele frequency due to selection
\begin{itemize}
\item Changes in population mean of a trait is due to the change in allele frequencies of the loci that affect that character
\item But often loci cannot be identified and changes in frequencies cannot be tracked
\item Can relate the effect of an allele on a quantitative trait using the expected change in allele frequency by selection
\item Allele frequencies change by natural selection according to the selective advantage factor $s$ as average fitness approaches , or $\Delta q = spq/2$
\item Selection coefficient against a genotype is $s = \frac{\alpha i}{\sigma_P}$ 
\item The above expression allows us to calculate the probability of fixating an allele with artificial selection using Kimura equation which is a function of s, $N_e$ and N
\item Given that the contribution to the mean of a locus is $M = \alpha Q$ 
\end{itemize}
\item Effect of selection on genetic variance
\begin{itemize}
\item the Bulmer effect: a transient decrease in genetic variance in the first few generations of selection. This reduction arises from the fact that selected individuals correspond to extremes, meaning reduced variance.
\item The reduced additive variance $V_A^* = V_A(1-kh^2)$
\item The interpretation of the reduced additive variance lies in negative LD, since selected individuals will be more extreme and similar. By virtue of carrying alleles that increase a trait, individuals also carry more alleles that reduce the trait, generating a negative LD with magnitude $-kh^2$. 
\item The disequilibrium is lost by half every generation, because every generation offspring are produced in full-sib families that have genetic correlations of 1/2. Half of genetic variance is within families and half is between families, and the total variance in the offspring $V_{AW} + V_{AB} = V_A(1-0.5kh^2)$ 
\end{itemize}
\item $N_e$ with artificial selection
\begin{itemize}
\item In selection scenarios, we can present $N_e$ as $\frac{N}{1+Q^2C}$, where $C^2$ is the variance of the selective advantage and $Q$ is the drag of the selective advantage
\item $G$ is the proportion of variance remaining after decay by genetic drift and selection, and ignoring drift, $G = 1-kh^2$
\item The asymptotic effect size with selection is half the number of breeders
\end{itemize}
\item Combined effect of genetic drift and selection on genetic variance
\begin{itemize}
\item The total genetic variance in generation t is equal to $V_{A,t} = V_{A,t-1}(1-\frac{1}{2N_e})$. This is called genic variance to distinguish from additive genetic variance in 9.11
\item Adding the reduction in variance by selection, the additive variance with selection per generation is obtained as $V_{A,t}^* = 0.5 V_{A,t-1}^*(1-1/N_e)(1-kh^2_{t-1})) + 0.5V_{A,t-1}$ 
\item Must be noted that $N_e$ calculated by these equations strictly corresponds to the generation for which it is calculated (since $N_e$ is reduced over generations)
\end{itemize}
\item Long term response of selection
\begin{itemize}
\item Short term response to selection only depends on intensity of selection, heritability and phenotypic variance
\item But in the long term, the response to selection is dependent on initial allele frequencies, effects of genes, effective population size and mutation. Response to selection is also dependent and greater with the number of loci that affect the trait segregating in the population
\item It is actually possible to make an estimate of the number of loci affecting a trait using the divergence between two lines from the same base population selected in opposite directions
\item If these lines were initially inbred, and each line is different for the trait, they are likely heterozygous. In the selection scenario, assuming high $N_e$ and no genetic drift, and that for all loci the difference between the two homozygotes is the same (a), then the total divergence reached at the selection limit is $R_T = na$. 
\item Initial additive variance of the cross: $\sigma_A^2 = \sum_n (a^2/2)(0.5)(0.5) = na^2/8$, yielding that the divergence is equal to n (in other words, if we know the magnitude of the response and the initial additive variance we can estimate the number of loci, termed the effective number of loci)
\item The effective number of loci is typically an underestimation if effects are not equal andif alleles are not fixed (or an overestimate if the initial frequencies are not intermediate)
\item This entire procedure of estimating the effective number of loci is the Castle-Wright method, and though generally the assumptions do not hold, it has some use in counting the number of loci with large effects in a cross 
\end{itemize}
\item Predicting long term response
\begin{itemize}
\item Using the infinitesimal model, we can obtain an estimate for the theoretical maximum response of selection over the long term
\item Ignoring mutation and the bulmer effect, any reductions in additive variance are exclusively from genetic drift 
\item The long term response to selection can then be approximated by $R_t = 2N_eR_1F_t$, and that over many generations the cumulative response in the limit is $2N_e$ times the initial response (implying additivity)
\item Antagonism between short and long term response: in the first generation, response depends on selection intensity and can be increased by reducing the number of parents and effective size (restricting long-term response)
\item The maximum final response over the long term will occur when a proportion of $phi = 0.5$ is selected
\item half-life of a response: the number of generations elapsed for the cumulative response to be half the response in the limit can be approximated by $R_t \approx 2N_eR_1[1-e^{-t/2N_e}]$ 
\item Impact of mutation: for a given trait, population and selection intensity, the response to selection is proportional to the additive variance in the mutation-drift equilibrium, given by $R_{M1,\inf} = i\sigma_{A,\inf}^2/\sigma_P$
\item The above is also the asymptotic response due to new mutations 
\item The cumulative response to selection hits a plateau once the initial additive variance is exhausted, ad further responses are only due to new mutations (becoming linear) 
\end{itemize}
\item Infinitesimal model versus empirical data
\begin{itemize}
\item The greater the leptokurtosis of the distribution, the lower the response will be at the limit than that predicted by the infinitesimal model
\item The response can also be irregular with high variation between replicates of the selected line
\item Utility of the infinitesimal model: does well for theoretical predictions about the impact of an initial bottleneck
\item Reduction of selection response with linkage: pronounced only if genome size is very small, in which case the infinitesimal model struggles to hold up
\item More problems: phenotypic variance can increase as generations go on, and this can be somewhat corrected by calculating the coefficient of phenotypic variation (std. dev divided by the mean) but even so, the coefficient increases drastically. A possible explanation is that homozygous genotypes have higher environmental sensitivity and/or genotype-environment interactions 
\item Selection can increase environmental sensitivity since the phenotypic of individuals with greater sensitivity will increase in better environments, and those are the individuals that are selected, increasing the sensitivity of the population as a whole
\item Illinois experiment: sought to increase and decrease the percentage of protein and fat in corn grain for more than 100 generations, showed no limit to selection response
\item Reasons for limits to selection response: depletion of genetic variation by fixation of selected alleles, or without exhausting genetic variation: favourable genes might be dominant in which case selection leads to high frequencies but will not but fixed, or overdominant genes can contribute null additive variance (and again no selection response)
\item Most common reason for selection limit: natural selection is opposed to artificial selection. t is likely that lethal or deleterious genes with pleiotropic effects increase in frequency with selection 
\item Ways to quantify and work around the above problem: compare the selection differential weighted by the number of offspring compared to the unweighted value, or we can relax selection and see if this change is lost by natural selection 
\end{itemize}
\item Family, within-family and BLUP selection
\begin{itemize}
\item Sometimes it is more prudent to carry out selection on entire families rather than individuals, or within families (when heritability is low, for milk production or fertility)
\item Selection criterion now is not the phenotype but the family mean, the deviation of the individual from the family mean, or a combined phenotypic value of individual and relatives, which generalizes to the equation $R = ir_{CA}\sigma_A$ where $r_{CA}$ is the accuracy of selection criterion (calculated as the correlation between applied criterion C and the additive value A of an individual). For individual selection, this value is simply the square root of heritability
\item Conducting within-family selection: take the best male and female of each family, then compute C as the phenotypic value in relation to other individuals of the same sex
\item Within-family deviations: another method where each individual is selected based on the deviation of phenotypic value from average of family (here more than one individual from each sex can be selected)
\item Within family versus within family deviations: get the expression $r_{P_W,A} = \frac{h(1-r)}{\sqrt{1-t}}$. For full-sib families, $r = 0.5$, so the accuracy of within-family selection is half that of individual selection 
\item If t is high the accuracy of within-family selection improves, and this is the advised method when there is high environmental variation common to families 
\item Selecting complete families: the accuracy of family selection is half that of individual selection but increases when n is high and t is low, or when heritability and c-squared is low (meaning that the noise from low heritability is compensated with the average values of individuals in the family). In short, family selection is good when heritability and common environmental variation are low.
\item Selection indices: increase selection response by using a combination of within- and between- family information. Example index: $I = b_wP_W + b_FP_F$ tracking deviations with a family and between families to the general mean 
\item Getting coefficients to optimize response: maximize selection accuracy by minimizing sum of squares of linear regression of index I on additive values 
\item In doing the above, family selection is good when $t << r$, within-family selection is good when $t>>r$ and individual selection is good when $t~r$ 
\item Optimal index will always be achieved using BLUP which uses information in the matrix of additive relationships of individuals from all generations (good because we account for genetic changes across generations and fixed effects)
\item Comparisons of all methods thus fart: optimized selection yields maximal response for short term, family response is better than individual initially if t is low but then drops sharply, and within-family selection yields a better response than family selection
\item Long term response depends on accuracy, additive variance and $N_e$. So family selection implies a big decrease in $N_e$ because entire families are lost, within-family equalizes family contributions, so effective size is equal to 2n breeding individuals. Individual, index and BLUP selection yields effective sizes intermediate to the previous 2. 
\item Long term response of within-family selection is greater than others: family selection is much lower
\item Selection of multiple traits: use selection indices, but again more precise using BLUP
\end{itemize}
\item Use of molecular markers
\begin{itemize}
\item Marker assisted selection: additive effects of markers associated with a trait are condensed into a molecular value for each individual, treat this as a character correlated with a trait of interest
\item MAS can be applied to BLUP yielding marker-assisted BLUP: good when phenotypic information is not available 
\item Success of marker-assisted selection is modest: association between marker and loci with an actual effect can be spurious and is lost by recombination. Extra response is reduced with generations of selection because variance of marker-assisted loci decrease as allele frequency decreases. Long-term response can also be reduced if selection decreases $N_e$. 
\end{itemize}
\end{itemize}

{\large \bf Papers}
Ideally empirical
\begin{itemize}
\item Vilas et al 2015: Allelic diversity for neutral markers retains a high adpative potential for quantiative traits than expected heterozygosity
\item Dudley and Lambert 2004: 100 generations of selection for oil and protein in corn
\item Yoo 1980: Long-term selection for a quantitative trait in large replicate populations of Drosophila melanogaster
\item LAnder and Botstein 1989: deduce the presence or absence of loci with large effects 
\item Lopez-Fanjul 1977: optimal selected proportion to produce a maximum response after a certain number of generations
\item Hill and Robertson 1966: selection with infinitesimal model under linked loci 
\item Long-term selection: Goodale 1938 until Wilson et al. 1971
\item Wilson 1971: increase in phenotypic variation over generations of selection 
\item Mulder 2008, Jill and Mulder 2011: reduce phenotypic variation between individuals by selecting for homogeneity given genetic variation for residual variation 
\item Hill 1996: compare within-family selection and within-family deviations 
\end{itemize}

{\large \bf Problems}   
\begin{enumerate}
\item The aim is to carry out an experiment to estimate the heritability of a trait using pairs of parents and offspring. It is expected that the estimate of heritability will be around 0.6. It is intended to evaluate only one offspring per couple and a single parent or both. How many pairs of data would have to be evaluated in each case to obtain an estimate of heritability with a standard error equal to or less than 0.05?
\begin{gather*}
SE(h^2) = \frac{2}{\sqrt{n}} \\
n = (\frac{2}{SE(h^2)})^2 = (\frac{2}{0.05})^2 = 1600\\
SE(h^2) = \sqrt{\frac{2}{n}}\\
n = (\frac{\sqrt{2}}{SE(h^2)})^2 = (\frac{\sqrt{2}}{0.05})^2 = 800
\end{gather*}

\end{enumerate}

{\large \bf Self Assessment}
\begin{enumerate}
\item The estimate of heritability obtained from the regression of offspring values on that of their parents will necessarily be biased if the latter are not a random sample of the population.\\
False, if the group of parents were selected, the regerssion estimates would not be affected if the bias would occur to the same extent in the numerator and the denominator of the equation $b_{OP} = \frac{cov(O,P)}{\sigma^2_P}$\\

\end{enumerate}


\medskip



\end{document}