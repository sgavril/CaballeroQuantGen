\documentclass[12pt]{amsart}

\addtolength{\hoffset}{-2.25cm}
\addtolength{\textwidth}{4.5cm}
\addtolength{\voffset}{-2.5cm}
\addtolength{\textheight}{5cm}
\setlength{\parskip}{0pt}
\setlength{\parindent}{15pt}

\usepackage{amsthm}
\usepackage{amsmath}
\usepackage{amssymb}
\usepackage[colorlinks = true, linkcolor = black, citecolor = black, final]{hyperref}

\usepackage{graphicx}
\usepackage{multicol}
\usepackage{ marvosym }
\usepackage{wasysym}
\newcommand{\ds}{\displaystyle}


\pagestyle{myheadings}

\setlength{\parindent}{0in}

\pagestyle{empty}

\begin{document}

\thispagestyle{empty}

{\scshape VETM 690} \hfill {\scshape \Large  Ch. 10: Natural Selection} \hfill {\scshape March 28 2021}
\medskip
\hrule
\bigskip

{\large \bf Quantitative traits and fitness:}
\begin{itemize}
\item Natural selection is the main force for changes in allele frequencies
\item Only selection produces changes that promote adaptation
\item Natural selection acts directly on a single character or fitness and the changes that occur between different quantitative traits depend on the genetic correlation between those traits and fitness
\item Quantitative traits can be represented as a pyramid where the top is fitness and main components are immediately below (viability, fecundity, mating success) and then a large number of interrelated traits through multiple and overlapping connections 
\item Traits at the bottom of the pyramid are neutral and loosely related to fitness, and consequently high heritabilities and low or no non-additive genetic variance and inbreeding depression
\item Traits at the tip of the pyramid show high inbreeding depression, low heritabilities due to environmental contributions and non-additive genetic components
\item Modelling response to natural selection: similar to previous chapter
\begin{itemize}
\item Take a population with census size $N$, and the contribution of individual $i$ is $W_i$
\item $\bar{W}$ is the population mean, and the selection differential is the average superiority of selected parents weighted by their relative contribution to the next generation
\item Or, symbolically, $\sigma_{i=1}^N(W_i-\bar{W})W_iN = (\sigma_{i=1}^NW_i^2/N) - \bar{W^2} = \sigma^2_{P,W}$ 
\item The selection differential is the phenotypic variance for fitness, and since we have $R = h^2S$, then $R_W = \sigma^2_{A,W}$ or the response to natural selection is equal to the additive variance for fitness (fundamental theorem of natural selection)
\item So natural selection will produce response until maximal fitness where additive variance becomes 0. But mutation implies that additive variance is constantly generated for fitness  (so selection continuously acts in a circle to reduce that same variation)
\item Scaled by resource availability, selection acts to maintain the status quo.
\item But new equilibria can appear depending on mutations, the external environment and other species 
\item Soft selection: a type of selection where census size remains constant and fitness is dependent on the density and frequency of individuals 
\item Hard selection: when operation of selection implies an increase or reduction in census size due to birth and mortality of individuals or limitation of resources
\item Non-additive variance (dominance, epistatic) may not be neglible for fitness components 
\item Even without the above, there could be additive variance for main components even if population fitness is maximum
\item It is common to observe antagonistic pleotriopy generating negative correlations between fitness components (one locus affects more than 1 trait in opposite directions). Ex: increase in fecundity may decrease viability
\item correlated selection differential for a trait X as a result of selection on fitness $S_{CX} = cov_P(X,W)$, and in turn the correlated response will be $R_{CX} = cov_A(X,W)$ (second theorem of natural selection)
\item Selection gradient: the ratio between the correlated selection differential and phenotypic variance in the regression $b_{P(W.X)} = S_{CX} / \sigma^2_{PX}$
\item Generalization of the effect of multiple quantitaive traits on the response to natural selection: $R_{CX_i} = b_{P(W.X_i)}\sigma^2_{A,X_i} + \sigma_{j \neq 1}^nb_{P(W.X_j)}cov_A(X_i,X_j)$ or more succintly, $R = Gb$ 
\item Selection gradiants can be obtained by estimating the partial regressions of relative fitness on the measures of the trait
\item Example using Darwin's finches: selected for shorter and wider beaks based on survival as a measure of fitness before and after a strong drought
\end{itemize}
\item Directional, stabilizing and diversifying selection
\begin{itemize}
\item  Negative selection: natural selection eliminates deleterious mutations for main components of fitness
\item At the same time, this increases the frequency of beneficial mutations until fixation, generating evolutionary novelties (positive selection)
\item For traits that are less strongly related to fitness: find stabilizing selection (individuals with intermediate phenotypes are favoured)
\item Diversifying selection: mutation favours more extreme individuals 
\item Convergent selection: when two or more populations are subject to stabilizing selection for the same optimum, and the genetic differentiation between them for that trait is smaller than expected (using a neutral model) 
\end{itemize}
\item Intensity of real and apparent stabilizing selection
\begin{itemize}
\item Fitness function: probability of survival conditioned to the phenotype. Not a density function! Amplitude to the point of inflection is the measure of the standard deviation of the normal distribution and is quantified as $\sqrt{V_S}$ where $V_S$ is an inverse measure of the intensity of stabilizing selection 
\item Mean population fitness can be obtained by integrating fitness weighted by trait density: $\bar{W} = \sqrt{\frac{V_PV_S}{V_P + V_S}}$
\item Variance of the distribution after selection: $V_P^* = V_P - \frac{V_P^2}{V_P + V_S}$
\item Stabilizing selection may be not be apparent if the trait is not subject to stabilizing selection but its assocliated loci have pleiotropic effects on fitness (requires a high  correlation between the trait and fitness)
\item Assuming a pleiotropic effect on fitness in heterozygosis of $sh$, and recalling that the genetic variance of a trait at mutation-selection balance $V_G = V_M/sh$, then it follows that $sh = V_M/V_G$. Substituting in common values, the apparent stabilizing selection by pleiotropy appears to be very weak.
\end{itemize}
\item Selection in heterogenous environments
\begin{itemize}
\item If populations reside in areas with environmental heterogeneity, the adaptation of individuals to different habitats can result in quantitative trait differentaition
\item If reproductive isolation is achieved, this can lead to speciation
\item In text example: high coast which has insolation, less waves and crabs led to molluscs with ribbed banded ecotypes while low coast which has humidity, strong waves and no crabs led to smooth unbanded ecotypes.  
\item Possible to estimate heritability in mollusc example because embryos of the same female are kept in the pouch before laying and shells are already formed. Comparing $Q_{ST}$ with $F_{ST}$ for neutral markers can test for diversifying selection
\item Different mollusc ecotypes had similar heritabilities, but the genetic differentiation between ecotypes was much greater than that observed between populations of the same ecotype in different transects 
\item Review: $Q_{ST}$ is a dimensionless parameter that is analagous to $F_{ST}$. The former provides an average of the genetic divergence for a trait as a function of total variation. The latter constitutes a measure of allele frequency differentiation between populations.
\item In the molluscs, genetic differentiation between ecotypes was much greater than genetic differentiation within the same ecotype in different transects. Similarly, $F_{ST}$ for these molluscs (the allele frequency differentiation for neutral markers) was similar to $Q_{ST}$ between transects (or within an ecotype) but lower than the value of $Q_{ST}$ between ecoptypes (these suggest that diversifying selection has promoted divergence between ecotypes).
\end{itemize}
\item Genetic variance and natural selection
\begin{itemize}
\item Natural selection will decrease genetic variance (except for diversifying selection)
\item Reductions in genetic variance by natural selection and drift are counterbalanced by mutation
\item Excess additive genetic variation has been observed in nature, and explanations for this include genotype-environment interaction and antagonistic pleiotropic effects of mutations on fitness components that are different from viability
\item Neutral characters have been shown to be a direct function of $N_e$, and should theoretically increase with census size. But this has led to Lewontin's paradox: repeated observation of neutral genetic diversity that ranges by an order of magnitude when population sizes differ by more than 3 orders of magnitude
\item Explanations for Lewontin's paradox: effective size does not increase proportionally to census size (species with large census sizes have effective sizes that are proportionally smaller than those with small census sizes). Two possible reasons for this, 1): species with large census size experience large demographic disturbances (high variation in reproductive success and oscillations in census size) which reduces $N_e$ and 2) effective size can be reduced due to constriction of genetic variation as a consequence of natural selection (either by purifying selection or selection of beneficial variants that greatly reduce neutral recombination)
\item In short, though we would expect neutral genetic variation to increase with $N_e$, we can see that constriction of this value by demographic and selective causes implies that differences in genetic variances between species with of large or small census size are not that large
\item Further note: neutral character traits can have deleterious pleiotropic effects on fitness. Genetic variance does not grow indefinitely in proportion to effective size (rather an asymptotic limit is reached) 
\item Traits subjected to stabilizing selection: two models for the genetic variance at equilibrium in infinite populations
\item Kimura-LAnde model: assume loci with high mutation rates per locus and infinite number of alleles per locus of small effect versus existing variation. At equilibrium, genetic variation is approximated by $\hat{V_G} = \sqrt{2nV_MV_S}$
\item House of cards model: low mutation rates and allelic effects of such large effect relative to stabilizing selection that mutations remain at low frequency, and stabilizing selection only results in selection against rare alleles 
\item Named the house of cards model because each mutation disturbs the existing equilibrium 
\end{itemize}
\item Genomic footprint of natural selection
\begin{itemize}
\item Previously saw that the impact of selection on $N_e$ is more intense with recombination is low
\item Genomically, selection can be studied by evaluating diversity at different points along the genome
\item Background/purifying selection reduces neutral variation in a homogenous manner 
\item Selection for beneficial alleles is more conspicuous: the rapid increase in the frequency of alleles produces a selective sweep (nearby neutral or deleterious alleles, hitchhikers) are also fixed with the beneficial allele. The lower the recombination rate, the greater the dragged portion.
\end{itemize}
\end{itemize}

{\large \bf Problems}   
\begin{enumerate}
\item In a field experiment, the partial regression of the individuals' fitness (W, scaled to the population mean) on the phenotypic values of two traits (X and  Y) was estimated, obtaining the selection gradients $b_{P(W.X)} = 0.8, b_{P(W.Y)}=0.1$. There are also estimates of the additive genetic variance of the two traits and their genetic correlation: $\sigma^2_{A,X} = 20, \sigma^2_{A,Y} = 30, r_A = 0.41$. (a) What are the expected correlated responses in the traits by the action of natural selection? (b) What would those responses be if the genetic correlation between the two traits were of the same magnitude but negative?
\begin{gather*}
R_{CX} = b_{P(W.X)}\sigma^2_{A,X} + b_{P(W.Y)}cov_A(X,Y)\\
R_{CX} = 0.8(20) + (0.1*10) = 17\\ 
R_{CY} = 0.1(30) + 0.8(10) = 11\\
R_{CX} = 0.8(20) + 0.1(-10) = 15\\
R_{CY} = 0.1(30) + 0.8(-10) = -5\\
\end{gather*}

\item The phenotypic variance for a character is $V_P = 30$ and that character is subject to stabilizing selection with intensity $V_S = 20$. By how much will the phenotypic variance will be reduced by the effect of selection? What decline in fitness is expected for each unit of increase in the square of the phenotypic values?
\begin{gather*}
V_{P*} = \frac{V_PV_S}{V_P + V_S} = (30*20)/(30+20) = 12\\
\Delta V_P = \frac{V_{P*}-V_P}{V_P} = \frac{12 - 30}{30} = -18/30 = -3/5 = -0.6\\
b_{w.y^2} = -1/(2V_S) = -1/40 = -0.025
\end{gather*}

\end{enumerate}

{\large \bf Self Assessment}
\begin{enumerate}
\item According to the Fundamental Theorem of Natural Selection, the genetic variance for the components of fitness is zero.\\
False, the fundamental theorem of natural selection states that the response to natural selection is equal to the additive variance for fitness, and not that the genetic variance of fitness components is zero.\\
\item Only neutral traits show a null genetic covariance with fitness.\\
False, if the relationship between a non-neutral trait and fitness is not linear, then the covariance between that trait and fitness can also be zero or null.\\
\item Stabilizing selection, by favouring individuals with intermediate phenotype, necessarily implies overdominance gene action for fitness.\\
False, stated on page 239 and though stabilizing selection implies the advantage of individuals with intermediate phenotypes, it does not necessarily imply overdominance. \\
\item Convergent selection implies stabilizing selection for the same optimum in two or more populations.\\
False, convergent selection implies that stabilizing selection for the same optimum in two or more populations results in genetic differentiation between the two populations is lower than expected versus a neutral model.\\
\item The greater the intensity of stabilizing selection, the greater the value of $V_S$, a measure of the width of the Gaussian curve that determines fitness.\\
False, $V_S$ is an inverse measure of the strength of stabilizing selection so lower $V_S$ means greater intensity.\\
\item The intensity of apparent stabilizing selection on a trait, due to pleiotropic effects on fitness, depends on the deleterious effect of mutations in heterozygosis.\\
This is true, it is stated that the intensity of apparent stabilizing effect due to pleiotroptic effects on fitness depends on the genetic variance of a trait at the mutation-selection balance. In 7.2.2, it is mentioned that deleterious alleles in heterozyogsis produce twice the mutation load compared to homozygotes.\\
\item A value of quantitative genetic differentiation $Q_{ST}$, greater than the corresponding value of differentiation in allele frequencies for neutral markers $F_{ST}$ allows diversifying selection to be inferred.\\
This is true, in chapter 8 it was mentioned that under selection, dominance accentuates the increase in $Q_T$ relative to $F_{ST}$ facilitating the detection of diversifying selection.\\
\item For a purely neutral character, it is predicted that the genetic variance increases linearly with the effective population size.\\
This is true, but not observed in practice due to demographic disturbances and constriction in genetic variance. \\
\item All types of natural selection reduce genetic variance.\\
False, diversifying selection increases genetic variance.\\
\item Selective sweeps of beneficial alleles reduce the neutral genetic diversity near the selective locus, but background selection can increase it.\\
This is false. It is true that selective sweeps of beneficial alleles can drag along neutral alleles, producing drastic reductions in diversity in areas close to the locus, but background selection will not be able to increase this diversity. If anything background selection will also decrease genetic diversity across the entire genome.\\
\medskip
\end{enumerate}
\end{document}