\documentclass[12pt]{amsart}

\addtolength{\hoffset}{-2.25cm}
\addtolength{\textwidth}{4.5cm}
\addtolength{\voffset}{-2.5cm}
\addtolength{\textheight}{5cm}
\setlength{\parskip}{0pt}
\setlength{\parindent}{15pt}

\usepackage{amsthm}
\usepackage{amsmath}
\usepackage{amssymb}
\usepackage[colorlinks = true, linkcolor = black, citecolor = black, final]{hyperref}

\usepackage{graphicx}
\usepackage{multicol}
\usepackage{ marvosym }
\usepackage{wasysym}
\newcommand{\ds}{\displaystyle}


\pagestyle{myheadings}

\setlength{\parindent}{0in}

\pagestyle{empty}

\begin{document}

\thispagestyle{empty}

{\scshape VETM 690} \hfill {\scshape \Large  Ch. 5: Effective Population Size} \hfill {\scshape Feb 15 2021}
 
\medskip

\hrule

\bigskip

\bigskip

Test test test

\bigskip

{\large \bf Notes:}

\begin{itemize}
\item  test

\end{itemize}

\smallskip


\medskip

{\large \bf Problems}   
\begin{enumerate}
\item In problem 4.1 we considered a set of ideal populations of census size N = 20 individuals each. Suppose now that the populations are maintained with 4 males and 16 females each generation. What would be the expected values of the heterozygosity, the variance in the allele frequencies and the inbreeding at generations $t = 5, 20$ and $100$?

Test answer

\item In problem 4.2 we considered a plant species whose natural reproduction is a combination of autogamy and allogamy in equal proportions. From a large population of this species a line with census size $N = 20$ was founded, which was maintained for 10 generations in the greenhouse by allogamous reproduction. 
\end{enumerate}


{\large \bf Self Assessment}


\medskip



\end{document}